%%%%%%%%%%%%%%%%%%%%%%%%%%%%%%%%%%%%%%%
% Modern Cover Letter Template (Deedy-style)
% 
% INSTRUCTIONS:
% 1. Copy this file to src/applications/
% 2. Rename: YYYY-MM-DD_CompanyName_Position.tex
% 3. Replace all {{PLACEHOLDERS}} with your content
% 4. Compile: ./compile.sh your_file.tex
%%%%%%%%%%%%%%%%%%%%%%%%%%%%%%%%%%%%%%%
\documentclass[10pt]{article}
\usepackage{fontspec}
\usepackage{geometry}
\usepackage{fancyhdr}
\usepackage{xcolor}
\usepackage{graphicx}
\usepackage{hyperref}
\usepackage{tikz}
\usetikzlibrary{shadows}
\usepackage{parskip}
\usepackage{enumitem}
\usepackage{tabularx}
\usepackage{fontawesome5}

\geometry{left=18mm,right=18mm,top=12mm,bottom=12mm}
\definecolor{accent}{HTML}{0B4671}
\definecolor{muted}{HTML}{666666}
\definecolor{linkedin}{HTML}{0A66C2}

% Hyperref settings: remove boxed links and use colored links. Keep general URLs muted;
% individual links (e.g., LinkedIn) can be explicitly colored where desired.
\hypersetup{
  colorlinks=true,
  linkcolor=accent,
  citecolor=accent,
  filecolor=accent,
  urlcolor=muted
}

% Font selection: the parser can inject a custom \setmainfont call via the
% {{MAIN_FONT_SETUP}} placeholder. If no font is provided via the parser,
% the parser will insert the default selection logic here.
{{MAIN_FONT_SETUP}}

% Slightly increase base line spacing for readability
\linespread{1.03}

% Profile picture size & macro with subtle drop shadow
\newcommand{\profilepicsize}{2.6cm}
\newcommand{\profilepic}[1]{%
  	\tikz[baseline=(img.base)] \node[inner sep=0pt,rounded corners=2pt,drop shadow] (img) {\includegraphics[width=\profilepicsize,keepaspectratio]{#1}};%
}

% Header/footer
\pagestyle{fancy}
\fancyhf{}
% Page numbers removed for single-page cover letters
\rfoot{}
\renewcommand{\headrulewidth}{0pt}

% Simple Deedy-like macros
\newcommand{\namesection}[3]{%
  \begin{minipage}[t]{0.72\textwidth}
    {\LARGE\bfseries #1~#2}\\
    {\small #3}
  \end{minipage}\hfill
}

\newcommand{\companyname}[1]{\textbf{#1}\\}
\newcommand{\companyaddress}[1]{#1}
\newcommand{\currentdate}[1]{\raggedleft #1}
\newcommand{\lettercontent}[1]{\par #1 \par}
\newcommand{\closing}[1]{\par #1}
\newcommand{\signature}[1]{\par #1}

% Signature block: prefer an image (e.g. figures/signature.png) and fall
% back to a printed typed name and title. Templates should call this with
% three args: path-to-image, printed-name, printed-title. The parser
% will replace common signature filenames with a user-provided
% `signature_image` path from `user_info.yml` when available.
\newcommand{\signatureblock}[3]{%
  % Simple left-aligned signature insertion. Avoid enclosing boxes or
  % minipages so the signature prints immediately after the preceding
  % paragraph/closing.
  \noindent
  % If an image exists at the provided path, render it; otherwise leave
  % a consistent vertical gap so layout doesn't jump.
  \IfFileExists{#1}{%
    \includegraphics[height=1.2cm]{#1}\par\vspace{4pt}
  }{%
    \vspace{1.2cm}\par\vspace{4pt}
  }%
  {\small\textbf{#2}\\#3}%
}

\begin{document}

% ============================================================
% HEADER - Your contact information
% ============================================================
\noindent
\begin{minipage}[t]{0.06\textwidth}
  \colorbox{accent}{\parbox[c][0.12\textheight][c]{\linewidth}{}}
\end{minipage}\hfill
\begin{minipage}[t]{0.72\textwidth}
  {\LARGE\bfseries {{YOUR_FULL_NAME}}}\\
  {\small {{YOUR_JOB_TITLE}}}\\
  {\small\color{muted}{{YOUR_TAGLINE_OR_EDUCATION}}}
\end{minipage}\hfill
\begin{minipage}[t]{0.18\textwidth}
  \raggedleft
  % Try several possible profile picture filenames (PNG/JPG/JPEG or named files)
  % Parser normalizes ../figures/ -> figures/ so reference figures/ here.
  \IfFileExists{figures/profile-pic.png}{%
    \profilepic{figures/profile-pic.png}\\[1pt]
  }{%
    \IfFileExists{figures/profile-pic.jpg}{%
      \profilepic{figures/profile-pic.jpg}\\[1pt]
    }{%
      \IfFileExists{figures/profile-pic.jpeg}{%
        \profilepic{figures/profile-pic.jpeg}\\[1pt]
      }{%
        \IfFileExists{figures/profile_placeholder.png}{%
          \profilepic{figures/profile_placeholder.png}\\[1pt]
        }{%
          % Fallback: empty box placeholder (keeps layout)
          \fbox{\rule{0pt}{\profilepicsize}\rule{\profilepicsize}{0pt}}\\[1pt]
        }%
      }%
    }%
  }%
  % close the minipage started for the profile picture area
  \end{minipage}
\noindent\rule{\linewidth}{0.5pt}\\[4pt]

% ============================================================
% CONTACT INFO \& DATE
% ============================================================
\begin{minipage}[t]{0.6\textwidth}
  \small
  % Use the main font (not monospace) and muted color for contact items.
  % \mbox prevents awkward line breaks inside the contact tokens.
  \faEnvelope\ \href{mailto:{{YOUR_EMAIL}}}{\mbox{\color{muted}{{{YOUR_EMAIL}}}}}\\
  \faPhone\ \href{tel:{{YOUR_PHONE_TEL}}}{\mbox{\color{muted}{{{YOUR_PHONE}}}}}\\
  % LinkedIn uses the branded color for emphasis but still the main font.
  \faLinkedin\ \href{{{YOUR_LINKEDIN_URL}}}{\mbox{\color{linkedin}{{{YOUR_LINKEDIN}}}}}
\end{minipage}\hfill
\begin{minipage}[t]{0.38\textwidth}
  \begin{flushright}
    \small\color{muted}{{DATE}}
  \end{flushright}
\end{minipage}

\vspace{8pt}

% ============================================================
% RECIPIENT INFO
% ============================================================
\begin{minipage}[t]{0.5\textwidth}
  % Use explicit placeholders: COMPANY_NAME and JOB_TITLE
  \companyname{{{COMPANY_NAME}}}\\
  % Render job title as a single-line bold block (no wrapping), without background
  \companyaddress{\mbox{\textbf{\large {{{JOB_TITLE}}}}}}
\end{minipage}\hfill
\begin{minipage}[t]{0.49\textwidth}
  % empty
\end{minipage}

\vspace{6pt}
% LETTER CONTENT - Replace with your text
% ============================================================
\lettercontent{Dear {{RECIPIENT_NAME}},}

\lettercontent{
{{PARAGRAPH_1_INTRODUCTION}}
}

\lettercontent{
{{PARAGRAPH_2_TECHNICAL_EXCELLENCE}}
}

\lettercontent{
{{PARAGRAPH_3_EXPERIENCE_AND_VALUE}}
}

\lettercontent{
{{PARAGRAPH_4_STRATEGIC_FIT}}
}

\lettercontent{
{{PARAGRAPH_5_CLOSING_STATEMENT}}
}

% ============================================================
% CLOSING & SIGNATURE
% ============================================================
\vspace{8pt}
\closing{{{CLOSING_SALUTATION}}}\par\nobreak
% Use the signatureblock macro. Templates reference a common signature file
% `figures/signature.png` so the parser can swap in a user-provided
% `signature_image` path from user_info.yml (see parse_md_to_tex.py).
\signatureblock{figures/signature.png}{{{YOUR_FULL_NAME}}}{{{YOUR_JOB_TITLE}}}

\end{document}
